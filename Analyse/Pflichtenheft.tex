\part{Einführung} 

Dieses Pflichtenheft gilt als Teil des Software-Engeneering II Beleges vom Sommersemester 2014.
Aufgabenstellung:
Entwickeln  Sie ein SW-System, das die Verwaltung der Daten für Belegarbeiten, die auch parallel laufen können. Neben der Erfassung sind auch weitere Anwendungsfälle wie zum Beispiel „archivieren von Daten“ zu realisieren.


\part{Auftrag}


\part{Abgrenzung des zu entwickelnden Systems}
Die Datenbankumgebung wird in Form einer Sybase Anwendung bereits vom Auftragsgeber gestellt,muss allerdings vom Auftragsnehmer selbständig mit Tabellen und Testdaten gefüllt werden.
 
\part{Ausgangssituation und Zielsetzung}


\part{Systemeinsatz, Systemumgebung}


\part{Benutzerschnittstellen}


\part{Funktionale Anforderungen}
\begin{itemize}
\item Login mit verschiedenen Berechtigungen(Projektgruppe/Dozent)
\item hierarchische Auswahl der Auflistung der Belege, der Gruppen, der Daten der Gruppenmitglieder für Dozenten
\item Erstellung neuer Belege & Zuteilung einer Anzahl an Gruppenslots(Cases)
\begin{itemize}
\item Zuteilung von Themen und dem Beleg aus einem bearbeitbaren Themenpool durch Dozent
\item Zuteilung von Rollen und dem Beleg aus einem bearbeitbaren Rollenpool durch Dozent
\end{itemize}
\item lesender/schreibender Zugriff des Dozenten auf sämtliche Daten
\item Generierung einer PDF-Datei mit entsprechenden Daten des Beleges
\item Ausgabe von Datensätzen nach Erfüllung einstellbarer Kriterien(Namen, Rollenverteilung)
\begin{itemize}
\item relevant für Suchfunktionen und Generierung von E-Mail-Addresslisten
\end{itemize}
\end{itemize}
\begin{itemize}
\item Erstellung neuer Gruppen auf Basis eines vorgegebenen Beleg-Erstlogins
\item tabellarische Auflistung der Gruppendaten aus Gruppenperspektive
\item Änderungsfunktionen der Gruppendaten aus Gruppenperspektive
\end{itemize}

Optional:
\begin{itemize}
\item Thunderbird-Schnittschnelle(direktes Öffnen)
\item druckbares Formular zur Benotung
\end{itemize}

\part{Qualitätsanforderungen}


\part{Rahmenbedingungen}
\begin{itemize}
\item Nutzung des hochschuleigenen Sybase-Servers
\item Anmeldung der Gruppe über einzelnes Login
\end{itemize}

\part{Fehlertoleranzmaßnahmen}


\part{Anforderungen an die Dokumentation}


\part{Abnahmekriterien}


\part{Glossar, Verzeichnisse, Anhang}

